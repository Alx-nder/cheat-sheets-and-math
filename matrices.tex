\documentclass{article}
\usepackage{amsmath}

\begin{document}

\title{Matrix operations}
\author{Tyreek Alexander}
\maketitle

\section{Matrix addition}
If $A$ and $B$ are matrices of the same dimension, their sum $C = A + B$ is obtained by adding corresponding elements:
\[
C = \begin{bmatrix}
a_{11} + b_{11} & a_{12} + b_{12} \\
a_{21} + b_{21} & a_{22} + b_{22}
\end{bmatrix}
\]

\section{Matrix multiplication}
If $A$ is an $m \times n$ matrix and $B$ is an $n \times p$ matrix, their product $C = AB$ is an $m \times p$ matrix:
\[
C = \begin{bmatrix}
c_{11} & c_{12} & \cdots & c_{1p} \\
c_{21} & c_{22} & \cdots & c_{2p} \\
\vdots & \vdots & \ddots & \vdots \\
c_{m1} & c_{m2} & \cdots & c_{mp}
\end{bmatrix}
\]
where
\[
c_{ij} = \sum_{k=1}^{n} a_{ik} b_{kj}
\]

\section{Transpose of a matrix}
The transpose of a matrix $A$ is denoted by $A^T$ and is obtained by swapping rows with columns:
\[
A^T = \begin{bmatrix}
a_{11} & a_{21} \\
a_{12} & a_{22}
\end{bmatrix}
\]

\section{Determinant of a matrix}
For a $2 \times 2$ matrix $A$, the determinant is given by:
\[
\det(A) = \begin{vmatrix}
a_{11} & a_{12} \\
a_{21} & a_{22}
\end{vmatrix} = a_{11}a_{22} - a_{12}a_{21}
\]

\section{Inverse of a matrix}
For a $2 \times 2$ matrix $A$, the inverse is given by:
\[
A^{-1} = \frac{1}{\det(A)} \begin{bmatrix}
a_{22} & -a_{12} \\
-a_{21} & a_{11}
\end{bmatrix}
\]
provided that $\det(A) \neq 0$.

\section{Trace of a matrix}
The trace of a matrix $A$ is the sum of its diagonal elements:
\[ 
\text{Trace of a matrix } A = \sum_{i=1}^{n} a_{ii} 
\]

\section{Eigenvalues and eigenvectors}
Eigenvalues $\lambda$ of a matrix $A$ are found by solving:
\[ 
\det(A - \lambda I) = 0 
\]

Eigenvectors $\mathbf{v}$ corresponding to eigenvalue $\lambda$ satisfy:
\[ 
A\mathbf{v} = \lambda \mathbf{v} 
\]

\section{Rank of a matrix}
The rank of a matrix $A$ is the maximum number of linearly independent rows or columns.

\section{Symmetric matrix}
A matrix $A$ is symmetric if:
\[ 
A = A^T 
\]

\section{Orthogonal matrix}
A matrix $A$ is orthogonal if:
\[ 
A^T A = AA^T = I 
\]

\section{Matrix operations}
\begin{align*}
    A+B &= B+A, \\
    (A+B)+C &= A+(B+C), \\
    c(A+B) &= cA+cB, \\
    c(dA) &= (cd)A, \\
    A+0 &= A, \\
    A+(-A) &= 0, \\
    1A &= A, \\
    c(A+B) &= cA+cB, \\
    A(B+C) &= AB+AC, \\
    (A+B)C &= AC+BC, \\
    A(BC) &= (AB)C, \\
    A(B^T) &= (A^T)B^T, \\
    (AB)^T &= B^TA^T, \\
    (AB)^{-1} &= B^{-1}A^{-1}, \\
    (A^T)^{-1} &= (A^{-1})^T, \\
    \det(AB) &= \det(A)\det(B), \\
    \det(A^T) &= \det(A), \\
    \det(A^{-1}) &= \frac{1}{\det(A)}, \\
    \text{Trace}(A+B) &= \text{Trace}(A) + \text{Trace}(B), \\
    \text{Trace}(AB) &= \text{Trace}(BA), \\
    \text{Trace}(A^T) &= \text{Trace}(A), \\
    \text{Trace}(A^{-1}) &= \frac{1}{\text{Trace}(A)}, \\
    \text{Rank}(A) &= \text{Rank}(A^T), \\
    \text{Rank}(A) &= \text{Rank}(A^TA), \\
    \text{Rank}(A) &= \text{Rank}(AA^T)
\end{align*}

\end{document}